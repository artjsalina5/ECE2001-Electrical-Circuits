% TEX compiler = luatex
% copyright arturo salinas-aguayo 2025
\documentclass[12pt]{article}

\usepackage{graphicx}
\usepackage{amsmath}
\usepackage{array}
\usepackage{amsfonts}
\usepackage{fancyhdr}
\usepackage{geometry}
\usepackage{circuitikz}
\usepackage{subfigure}
\usepackage{caption}
\usepackage{karnaugh-map}
\usepackage{bm}
\usepackage{float}

\geometry{letterpaper, margin=1in}
\graphicspath{ {../../images/} }

% Header and Footer
\pagestyle{fancy}
\fancyhf{}
\fancyhead[L]{ECE 2001 - Module 1: Circuit Theory}
\fancyhead[R]{\thepage}
\setlength{\headheight}{15pt}

\author{Arturo Salinas-Aguayo}
\title{Module 1: Circuit Theory}
% theorem set
\newtheorem{example}{Example}
% Example block environment
\newenvironment{examp}
{\vspace{0.5cm}
 \hrule
\vspace{0.5cm}
\begin{example}}
{\hrule
\vspace{0.5cm}
\end{example}}

\begin{document}
\newcommand{\closure}[2][3]{%
	{}\mkern#1mu\overline{\mkern-#1mu#2}}
\newcommand\ncoverline[1]{\mkern1mu\overline{\mkern-1mu#1\mkern-1mu}\mkern1mu}
% Title Page
\begin{titlepage}
	\centering
	\vspace*{3cm}
	\huge\textbf{Module 1: Circuit Theory }\\
	\vspace{5cm}
	\Large\textbf{Arturo Salinas-Aguayo}\\
	\normalsize
	ECE 2001 Electrical Circuits\\
	Dr. David J. Giblin, Section 331.660.701.810-1253\\
	Mechanical Engineering Department
	\vfill
	\includegraphics[scale=0.1]{uconnlogo}\\
	College of Engineering, University of Connecticut\\
	\scriptsize{Coded in \LaTeX}
	\vspace*{1cm}
\end{titlepage}
\section*{Basic Concepts}
\subsection*{Assumptions}
\begin{itemize}
	\item \textbf{Lumped Parameter System}

	      In a system that is small enough to make the assumption that Electrical
	      effects happen instantaneously through the system.
	      \begin{examp}
		      \[
			      \lambda>>L \]
		      \[
			      \lambda = \frac{c}{f} = \frac{3\times10^8 m/s}{60s^{-1}} = 5\times10^6m
		      \]
	      \end{examp}
	\item \textbf{Net Charge}

	      The net charge on every component in a system is always zero.
	\item \textbf{Magnetic Coupling}

	      There is no magnetic coupling between components in a system.
\end{itemize}
\subsection*{Charge, Voltage, and Current}
\begin{itemize}
	\item Charge is discrete
	      \[q = 1.602\times10^{-19}C\]
	\item \textbf{Charge is Bipolar}
\end{itemize}
\end{document}
